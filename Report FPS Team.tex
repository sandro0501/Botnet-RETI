\documentclass[a4paper]{report}
\usepackage{hyperref}
\usepackage[T1]{fontenc}
\usepackage[utf8]{inputenc}
\usepackage[english, italian]{babel}
\usepackage{lipsum} 
\usepackage{url} 
\usepackage[table,xcdraw]{xcolor}
\usepackage {amssymb}
\usepackage{listings}
\usepackage{graphicx}
\usepackage{booktabs}
\usepackage{tabularx}
\usepackage[export]{adjustbox}
\lstset{
	basicstyle=\ttfamily,
	mathescape
}
\usepackage{float}
\restylefloat{table}
\usepackage[normalem]{ulem}
\useunder{\uline}{\ul}{}
\maxdeadcycles=1000
\begin{document}
\author {Ferdinando D'Alessandro\\N86003933\and Pasquale Kevyn Carderopoli \\ N86003931\and Santolo Barretta\\N86003666}
\title {
	\begin{figure}[hp]
		\centerline{\includegraphics[scale=.04]{SimboloFedericoII}}
	\end{figure} Report sullo sviluppo e le prove di esecuzione di una botnet in Python }


\date{}
\maketitle

\tableofcontents

\chapter{Sviluppo della botnet}

\section{Architettura client server}

Iniziando a sviluppare la botnet la prima decisione che abbiamo preso è stata la gestione dell'architettura.
Una botnet è una rete composta da uno o più bot che inviano dati al botmaster, nel nostro caso che comunicano tramite socket TCP.\\
Abbiamo deciso che i bot sarebbero stati i client mentre il botmaster il server, questo per più ragioni:
\begin{itemize}
	\item Ci risultava la scelta più appropriata da un punto di vista ideologico: noi dovevamo avere in mano la situazione e quindi usare il server. Ci sembrava giusto che fossero i bot a contattare il server una volta reperite le informazioni e non il contrario.
	\item Anche se è stata una simulazione, in una situazione realistica sarebbe scomodo dover sapere necessariamente l'ip di tutti i bot, mentre un botmaster è probabile sia configurato su un ip statico facilmente raggiungibile da tutti i bot (Come poi è stato nella prova finale).
	\item Dato che il botmaster è uno soltanto mentre è possibile ci siano più bot, risulta più comodo far connettere più bot client a un server botmaster.
	\item Se avessimo scelto l'architettura opposta avremmo avuto il botmaster che faceva da client e avrebbe dovuto connettersi ogni volta ad un bot server diverso, ostacolando una facile estensione della rete di bot.
\end{itemize}

\section{Funzionalità implementate}
Abbiamo deciso di sviluppare il bot in modo che rispondesse in tempo reale alle richieste fatte dal botmaster, effettuate attraverso un menù che abbiamo sviluppato.\\
Per ottenere le informazioni che volevamo sul dispositivo vittima del bot abbiamo fatto uso di diverse librerie tra cui platform, os, subprocess e psutil.
Ad ogni opzione del menù quindi corrisponde un set di informazioni che il bot ci restituisce, in particolare:
\\
\begin{enumerate}
	
	\item \textbf{Informazioni sul sistema:}
	\begin{itemize}
		\item Nome e versione del sistema operativo installato;
		\item Nome dell'utente corrente;
		\item Architettura del dispositivo;
		\item Tempo di avvio del sistema.
	\end{itemize}
	
	\item \textbf{Informazioni sulla CPU:}
	\begin{itemize}
		\item Nome del modello dei processori;
		\item Numero di core fisici e logici;
		\item Frequenza di CPU massima, minima e corrente;
		\item Percentuale di uso della CPU.
	\end{itemize}
	
	\item \textbf{Informazioni sul disco e le sue partizioni:}
	\begin{itemize}
		\item Nome delle partizioni;
		\item Tipo di file system;
		\item Punto di montaggio;
		\item Dimensione, spazio libero ed occupato.
	\end{itemize}
	
	\item \textbf{Informazioni sulla scheda di rete e sulle interfacce di rete:}
	\begin{itemize}
		\item Indirizzo MAC della scheda di rete;
		\item Indirizzo IP e maschera di sottorete delle interfacce.
	\end{itemize}
	
	\item \textbf{Statistiche sulla rete:}
	\begin{itemize}
		\item Numero di byte inviati e ricevuti;
		\item Numero di pacchetti inviati e ricevuti;
		\item Numero di pacchetti persi in invio e ricezione;
		\item Numero di errori in invio e ricezione.
	\end{itemize}
	
\end{enumerate}

Abbiamo poi deciso di implementare una \underline{reverse shell}, cioè inviare dei comandi affinché vengano eseguiti sulla macchina del bot client.\\
In particolare per eseguire il comando a distanza abbiamo fatto uso dei metodi \textit{os.getoutput(...)} e \textit{os.check\_output(...)} passando il comando ottenuto da input dopo aver escluso casi di comandi particolari come il cambio della directory corrente (\textit{cd}).\\
Abbiamo sfruttato la reverse shell per ottenere informazioni sulla macchina più approfondite laddove le librerie esterne non bastavano e per navigare il file system alla ricerca di file che potessero risultarci appetibili, preferendolo come approccio rispetto ad uno più diretto come il download o la scansione automatica dell'intero file system, che potrebbe diventare un processo troppo pesante e lento su alcune macchine.\\
Tuttavia senza estensioni ulteriori potevamo solamente stampare e ricavare il contenuto di file di testo, attraverso comandi come more o cat; per questo prima della seconda prova abbiamo anche aggiunto la possibilità di scaricare dei file in modo da poter gestire qualsiasi tipo di file.\\

\section{Tecniche per rendere più solido lo scambio di messaggi}
Molto del tempo di sviluppo della botnet è stato mirato al renderla il più solida possibile, data la facile tendenza a problemi di comunicazione lavorando con le socket.
Purtroppo nonostante tutte le misure di precauzione prese, il programma ha dato comunque dei problemi ma il nostro lavoro ha permesso alla situazione di ristabilirsi seppur con la perdita di qualche dato.\\
Ricordiamo che la receive nelle socket in Python prende una dimensione massima di byte da ricevere, questione che ha portato dubbi e problemi riguardanti la dimensione del buffer scelta. Infatti ci è risultato subito chiaro che, qualsiasi fosse stata la dimensione scelta, alcuni dei messaggi inviati dal bot sarebbero risultati troppo grandi. Per questo abbiamo lavorato in due modi:
\begin{enumerate}
	\item Dapprima abbiamo implementato una manovra forzata per cui nel caso in cui si fosse notato un problema di bufferizzazione, avremmo forzato il server tramite un comando a fare una receive extra, in modo da svuotare il buffer;
	\item Per la seconda prova invece abbiamo lavorato ad un'automatizzazione di questo processo, facendo sì che il client mandasse prima il numero di receive che il server avrebbe dovuto fare (Calcolato a partire dalla dimensione del messaggio chiaramente) e poi il messaggio vero e proprio, permettendo così al server di ricevere in modo adeguato il messaggio.\\
	Questo scambio di messaggi si può vedere meglio dalla scansione effettuata con Wireshark che è analizzata a fine report.
\end{enumerate}
Infine per cercare di salvare situazioni disastrose che si possono verificare in uno scenario reale e che purtroppo si sono verificate anche nelle simulazioni di prova, abbiamo manipolato client e server affinché cerchino sempre di riconnettersi non appena la connessione è persa.

\chapter{Prove di esecuzione}

\section{Prima prova}

Durante la prima prova abbiamo trovato le seguenti informazioni sulla macchina:
\begin{enumerate}
	\item
	\begin{itemize}
		\item \textbf{Nome utente:} Ubuntu-20-04-LTS
		\item \textbf{Sistema operativo:} Linux
		\item \textbf{Versione:} \#58-Ubuntu SMP Thu Oct 13 08:03:55 UTC 2022
		\item \textbf{Release:} 5.15.0-52-generic
		\item \textbf{Architettura:} x86\_64
	\end{itemize}
	\item La macchina possiede\textbf{ 4 processori uguali}, le cui specifiche sono:
	\begin{itemize}
		\item \textbf{Nome del modello:} Intel(R) Core(TM) i7-8569U CPU @ 2.80GHz
		\item \textbf{Frequenza:} 2811.904 MhZ
		\item \textbf{Numero di core:} 4
		\item \textbf{Dimensione della memoria cache:} 8192 KB
	\end{itemize}
	\item La macchina è \textbf{attiva dal giorno 16/11/2022 alle ore 09:04:08}
	\item La memoria RAM ha una \textbf{dimensione di 3.83GB} e nell'istante in cui abbiamo tratto le informazioni aveva
	\begin{itemize}
		\item \textbf{Memoria disponibile:} 2.52GB
		\item \textbf{Memoria in uso:} 1004.59MB
		\item \textbf{Percentuale memoria in uso:} 34.3%
	\end{itemize}
	\item Il disco presenta 16 partizioni, di cui:
	\begin{itemize}
		\item Una partizione principale:
		\begin{itemize}
			\item \textbf{Nome:} /dev/sda3
			\item \textbf{Punto di mount:} / (root)
			\item \textbf{Tipo di file system:} ext4
			\item \textbf{Spazio occupato:} 13.17GB
			\item \textbf{Spazio libero:} 9.53GB
			\item \textbf{Percentuale spazio occupato:} 58.0%
		\end{itemize}
		\item Una duplicazione di quest'ultima usata da HunSpell per il funzionamento dello spell check, con \textbf{punto di mount} /var/snap/firefox/common/host-hunspell
		\item Una partizione di boot:
		\begin{itemize}
			\item \textbf{Nome:} /dev/sda2
			\item \textbf{Punto di mount:} /boot/efi
			\item \textbf{Dimensione:} 511.96MB
			\item \textbf{Spazio occupato:} 5.24MB
			\item \textbf{Spazio libero:} 506.73MB
			\item \textbf{Percentuale spazio occupato:} 1.0%
		\end{itemize}
		\item 13 partizioni di loop generate dall'installazione di snap, tutte con \textbf{punto di mount} /snap/...
	\end{itemize}
	\item La scheda di rete ha \textbf{Indirizzo MAC: 00:00:79:DC:8B:51:77:5B} e due interfacce di rete ethernet oltre quella di loopback:
	\begin{enumerate}
		\item 
		\begin{itemize}
			\item \textbf{Nome:} enp0s3
			\item \textbf{Indirizzo IP:} 10.0.2.15
			\item \textbf{Network mask: }255.255.255.0
			\item \textbf{Broadcast:} 10.0.2.255
		\end{itemize}
		\item 
		\begin{itemize}
			\item \textbf{Nome:} enp0s8
			\item \textbf{Indirizzo IP:} 192.168.1.114
			\item \textbf{Network mask: }255.255.255.0
			\item \textbf{Broadcast:} 192.168.1.255
		\end{itemize}
	\end{enumerate}
	\item Infine nell'istante in cui abbiamo tratto le informazioni abbiamo trovato queste statistiche di rete:
	\begin{itemize}
		\item \textbf{Totale byte inviati:} 73.59MB
		\item \textbf{Totale byte ricevuti:} 4.48MB
		\item \textbf{Totale pacchetti inviati:} 57646
		\item \textbf{Totale pacchetti ricevuti: }23828
		\item \textbf{Totale errori in ricezione pacchetti:} 0
		\item \textbf{Totale errori in invio pacchetti:} 0
		\item \textbf{Totale pacchetti in arrivo persi:} 19
		\item \textbf{Totale pacchetti in uscita persi:} 0
	\end{itemize}
	
	Inoltre navigando il file system siamo riusciti a trovare il \textbf{file nascosto .passwords} che in quel momento conteneva la \textbf{stringa \textit{"eCambiata"}}, anche se confrontandoci con altri colleghi pare il contenuto originale fosse un altro e che quello che abbiamo trovato fosse il risultato di uno scherzo di un collega.	

\end{enumerate}

\section{Seconda prova}
La seconda prova riguardava due macchine diverse:
\subsection{Prima macchina}
Ci era stato detto che la prima macchina sarebbe stata la stessa della prima prova, ma a giudicare dall'indirizzo fisico della scheda di rete differente, immaginiamo fosse una macchina virtuale clone di quella usata nella prima prova.
\\Per questo oltre l'\textbf{indirizzo MAC}, variato da 00:00:79:DC:8B:51:77:5B a \textbf{00:00:0D:07:91:5F:74:E4}, solo i dati dinamici sono risultati diversi, come:
\begin{enumerate}
	\item Il \textbf{tempo di avvio}, che inizialmente era il \textbf{giorno 13/12/2022 dalle ore 16:28:18}, ma in seguito a problemi tecnici che hanno portato al riavvio della macchina è diventato il \textbf{giorno 13/12/2022 dalle ore 17:23:41}.
	\item Le partizioni, rimaste 16 ma con delle variazioni tra le partizioni di loop generate da snap.
	\item L'interfaccia di rete enp0s8, che presentava come \textbf{indirizzo ip 192.168.1.188} invece che 192.168.1.114.
	\item Le \textbf{statistiche di rete} e sull'\textbf{uso della memoria RAM} che date le circostanze della seconda prova sono aumentate rispetto alla prima, arrivando a:
	\begin{itemize}
		\item \textbf{Memoria disponibile:} 2.52GB
		\item \textbf{Memoria in uso:} 1000.79MB
		\item \textbf{Percentuale memoria in uso:} 34.2%
		\item \textbf{Totale byte inviati:} 417.29MB
		\item \textbf{Totale byte ricevuti:} 287.54MB
		\item \textbf{Totale pacchetti inviati:} 316075
		\item \textbf{Totale pacchetti ricevuti:} 245933
	\end{itemize}

\end{enumerate}
Navigando il file system abbiamo trovato alcuni file di interesse:
\begin{itemize}
	\item Il file \textbf{.profile} all'interno della home (/alessio) che non abbiamo avuto la prontezza di stampare, ma essendo stato modificato l'ultima volta il 12 Novembre immaginiamo contenesse ciò che un normale file .profile contiene in un sistema Linux, cioè variabili di sistema relative all'utente.
	\item Un file\textbf{ .myPwd} nella home, contenente la stringa \textit{\textit{"SBAGLIATO! VEDI ALTROVE"}}, rimandando alla ricerca di un eventuale altro file contenente delle password, che abbiamo poi scoperto essere presente nella seconda macchina.
\end{itemize}

\subsection{Seconda macchina}
La seconda macchina, a giudicare dalle informazioni ritrovate, era una macchina virtuale ottenuta inizialmente clonando la prima macchina. Riportiamo le poche differenze importanti notate nella scheda di rete e nel file system, sorvolando sui vari dati dinamici come le statistiche riportate sopra per la prima macchina.
\begin{enumerate}
	\item \textbf{Indirizzo MAC della scheda di rete:} 00:00:1D:1A:D5:F6:EE:3C
	\item \textbf{Indirizzo IP dell'interfaccia enp0s8:} 192.168.1.224
\end{enumerate}
Nel file system i file trovati d'interesse sono:
\begin{enumerate}
	\item Il file \textbf{.myPwd}, rimasto invariato.
	\item Un file \textbf{.profiles} non presente sulla prima macchina, contenente la stringa \textit{"QUASI..."}
	\item Il file \textbf{.profile}, modificato per contenere la stringa\\\textit{"WHAT? I PUT MY PASSWORDS HERE!}
	
	 \textit{AlessioPassword123}
	 \\\textit{ThisIsMyBestOne!}
	 \\\textit{isThisAJoke?}
	 \\\textit{sad@123@ertre@ERTY!"} \\nascosta nel contenuto standard del file generato dal sistema.
	 
\end{enumerate}

\subsection{Considerazioni sui risultati}
Complessivamente siamo soddisfatti dei risultati delle prove perché nonostante ci siano stati dei problemi che hanno portato alla perdita di frammenti di informazioni, siamo riusciti comunque a salvare le informazioni più importanti attraverso le tecniche di recupero preparate.

\chapter{Analisi di frammenti di comunicazione ottenuti con Wireshark}
Vediamo un frammento di comunicazione, a partire dalla 3-way handshake dell'apertura della connessione fino alla ricezione delle informazioni sul sistema operativo.

\begin{figure}[hp]
	\centerline{\includegraphics[scale=.5]{WS1}}
\end{figure}

Prima di tutto abbiamo l'apertura della connessione con la 3-way handshake tipica di TCP, riconoscibile dai pacchetti con i flag [SYN], [SYN, ACK] e [ACK].
\\Poi per come funziona l'applicativo, la prima cosa che è effettuata automaticamente è l'invio della directory corrente dal client al server.

\begin{figure}[hp]
	\centerline{\includegraphics[scale=.5]{WS2}}
\end{figure}
\begin{enumerate}
	\setcounter{enumi}{2468}
	\item Pacchetto client->server in cui è mandata la dimensione del messaggio in numero di pacchetti che deve essere inviato (in questo caso la directory corrente), contiene "1".
	\item Pacchetto server->client di acknowledgement manuale (ed automatico dato il flag ACK) contenente la stringa "SIZE RICEVUTA".
	\item Pacchetto client->server di acknowledgement automatico generato da TCP ([ACK]).
	\item Pacchetto client->server contenente l'effettiva working directory desiderata dal server.
	\item Pacchetto server->client di acknowledgement ([ACK]).
\end{enumerate}
Adesso al gestore del server è presentato il menù con la possibilità di chiedere dati al client e vediamo il funzionamento della richiesta di dati sul sistema operativo.
\begin{figure}[ht!]
	\centerline{\includegraphics[scale=.5]{WS3}}
\end{figure}

\begin{enumerate}
	\setcounter{enumi}{2884}
	\item Pacchetto server->client in cui è mandato il comando desiderato, in questo caso "1" che corrisponde alla richiesta di informazioni sul sistema operativo.
	\item Pacchetto client->server contenente la size in numero di pacchetti del messaggio da inviare, in questo caso "1".
	\item Pacchetto server->client di acknowledgement manuale contenente "SIZE RICEVUTA".
	\item Pacchetto client->server contenente l'effettivo messaggio contenente le informazioni sul sistema operativo vittima, a cui seguirà l'ACK del server.
\end{enumerate}


\end{document}