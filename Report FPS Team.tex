\documentclass[a4paper]{report}
\usepackage{hyperref}
\usepackage[T1]{fontenc}
\usepackage[utf8]{inputenc}
\usepackage[english, italian]{babel}
\usepackage{lipsum} 
\usepackage{url} 
\usepackage[table,xcdraw]{xcolor}
\usepackage {amssymb}
\usepackage{listings}
\usepackage{graphicx}
\usepackage{booktabs}
\usepackage{tabularx}
\usepackage[export]{adjustbox}
\lstset{
	basicstyle=\ttfamily,
	mathescape
}
\usepackage{float}
\restylefloat{table}
\usepackage[normalem]{ulem}
\useunder{\uline}{\ul}{}
\maxdeadcycles=1000
\begin{document}
\author {Ferdinando D'Alessandro\\N86003933\and Pasquale Kevyn Carderopoli \\ N86003931\and Santolo Barretta\\N86003666}
\title {
	\begin{figure}[hp]
		\centerline{\includegraphics[scale=.04]{SimboloFedericoII}}
	\end{figure} Report sullo sviluppo e le prove di esecuzione di una botnet in Python }


\date{}
\maketitle

\tableofcontents

\chapter{Sviluppo della botnet}

\section{Architettura client server}

Iniziando a sviluppare la botnet la prima decisione che abbiamo preso è stata la gestione dell'architettura.
Una botnet è una rete composta da uno o più bot che inviano dati al botmaster.\\
Abbiamo deciso che i bot sarebbero stati i client mentre il botmaster il server, questo per più ragioni:
\begin{itemize}
	\item Ci risultava la scelta più appropriata da un punto di vista ideologico: noi dovevamo avere in mano la situazione e quindi usare il server. Ci sembrava giusto che fossero i bot a contattare il server una volta reperite le informazioni e non il contrario.
	\item Anche se è stata una simulazione, in una situazione realistica sarebbe scomodo dover sapere necessariamente l'ip di tutti i bot, mentre un botmaster è probabile sia configurato su un ip statico facilmente raggiungibile da tutti i bot (Come poi è stato nella prova finale).
	\item Dato che il botmaster è uno soltanto mentre è possibile ci siano più bot, risulta più comodo far connettere più bot client a un server botmaster.
	\item Se avessimo scelto l'architettura opposta avremmo avuto il botmaster che faceva da client e avrebbe dovuto connettersi ogni volta ad un bot server diverso, ostacolando una facile estensione della rete di bot.
\end{itemize}

\section{Funzionalità implementate}
Abbiamo deciso di sviluppare il bot in modo che rispondesse in tempo reale alle richieste fatte dal botmaster, effettuate attraverso un menù che abbiamo sviluppato.\\
Per ottenere le informazioni che volevamo sul dispositivo vittima del bot abbiamo fatto uso di diverse librerie tra cui platform, os, subprocess e psutil.
Ad ogni opzione del menù quindi corrisponde un set di informazioni che il bot ci restituisce, in particolare:
\\
\begin{enumerate}
	
	\item \textbf{Informazioni sul sistema:}
	\begin{itemize}
		\item Nome e versione del sistema operativo installato;\textit{\\platform.uname().system\\platform.uname().release\\platform.uname().version}
		\item Nome dell'utente corrente;\textit{\\platform.uname().node}
		\item Architettura del dispositivo;\textit{\\platform.uname().machine}
		\item Tempo di avvio del sistema.\textit{\\psutil.boot\_time()}
	\end{itemize}
	
	\item \textbf{Informazioni sulla CPU:}
	\begin{itemize}
		\item Nome del modello del processore;
		\item Numero di core fisici e logici;
		\item Frequenza di CPU massima, minima e corrente;
		\item Percentuale di uso della CPU.
	\end{itemize}
	
	\item \textbf{Informazioni sul disco e le sue partizioni:}
	\begin{itemize}
		\item Nome delle partizioni;
		\item Tipo di file system;
		\item Punto di montaggio;
		\item Dimensione, spazio libero ed occupato.
	\end{itemize}
	
	\item \textbf{Informazioni sulla scheda di rete e sulle interfacce di rete:}
	\begin{itemize}
		\item Indirizzo MAC della scheda di rete;
		\item Indirizzo IP e maschera di sottorete delle interfacce.
	\end{itemize}
	
	\item \textbf{Statistiche sulla rete:}
	\begin{itemize}
		\item Numero di byte inviati e ricevuti;
		\item Numero di pacchetti inviati e ricevuti;
		\item Numero di pacchetti persi in invio e ricezione;
		\item Numero di errori in invio e ricezione.
	\end{itemize}
	
\end{enumerate}

Parte in cui parlare della reverse shell, dei comandi e della tecnica (fallente) per sapere la size a priori 

\chapter{Prove di esecuzione}

\section{Prima prova}

\section{Seconda prova}

\end{document}